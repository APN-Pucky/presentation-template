\usepackage{polyglossia}
\setmainlanguage{english} 
\usepackage{pgfpages}



% Spracheinstellung
%\usepackage{polyglossia}
%\setmainlanguage{german}

% Load theme
%%%%%%%%%%%%%%%%%%%%%%%%%
% DEFAULT - Comment All %
%%%%%%%%%%%%%%%%%%%%%%%%%

%%%%%%%%%%%%%%%%%%%%%%%%%
% WWU STYLE FS TEMPLATE %
%%%%%%%%%%%%%%%%%%%%%%%%%
%%% Einstellungen zur richtigen Benutzung von wwustyle.sty
\usefonttheme{professionalfonts}
%\RequirePackage{wwustyle/meta-office-pro}
% Einstellungen der Schriftart (Meta Office Pro) für Text und Mathematik
% math**=sym angeben, damit auch diese Befehle die Schriftart Meta verwenden
\usepackage[mathrm=sym, mathit=sym, mathsf=sym, mathbf=sym]{unicode-math}
\IfFileExists{fonts/MetaOffcPro-Bold.ttf}{
\setmainfont{MetaOffcPro}[Path=fonts/,
Extension=.ttf,
UprightFont=*-Norm,
UprightFeatures={
	SmallCapsFont={MetaScOffcPro-Norm}
},
ItalicFont=*-Norm,
ItalicFeatures={
	SmallCapsFont={MetaScOffcPro-NormIta}
},
BoldFont=*-Bold,
BoldFeatures={
	SmallCapsFont={MetaScOffcPro-Bold}
},
BoldItalicFont=*-Bold,
BoldItalicFeatures={FakeSlant},
]
\setsansfont{MetaOffcPro}[Path=fonts/,
Extension=.ttf,
UprightFont=*-Norm,
UprightFeatures={
	SmallCapsFont={MetaScOffcPro-Norm}
},
ItalicFont=*-Norm,
ItalicFeatures={
	SmallCapsFont={MetaScOffcPro-NormIta}
},
BoldFont=*-Bold,
BoldFeatures={
	SmallCapsFont={MetaScOffcPro-Bold}
},
BoldItalicFont=*-Bold,
BoldItalicFeatures={FakeSlant},
]
}{}
%\setmathfont{Latin Modern Math}
%% Meta Office Pro für die Bereiche nutzen, für die Glyphen existieren
%\setmathfont{MetaOffcPro-Norm.ttf}[Path=fonts/,
%range=up/{greek,Greek,latin,Latin,num}]
%\setmathfont{MetaOffcPro-NormIta.ttf}[Path=fonts/,
%range=it/{greek,Greek,latin,Latin,num}]
%\setmathfont{MetaOffcPro-Bold.ttf}[Path=fonts/,
%range=bfup/{greek,Greek,latin,Latin,num}]
%\setmathfont{MetaOffcPro-Bold.ttf}[Path=fonts/, UprightFeatures={FakeSlant},
%range=bfit/{greek,Greek,latin,Latin,num}]
%% Symbole (leider enthält Meta Office Pro nicht das Symbol ∓)
%%\setmathfont{MetaOffcPro-Norm.ttf}[Path=fonts/]
%
%\setmathfont{MetaOffcPro-Norm.ttf}[Path=fonts/,
%	range={
%		"002B,    % +
%		"002D,    % -
%		"2212,    % −
%		\minus,   % s. https://github.com/wspr/unicode-math/issues/451
%		"00D7,    % ×
%		"00F7,    % ÷
%		"00B7,    % ·
%		"22C5,    % ⋅
%		"002F,    % /
%		"2044,    % ⁄
%		"00B1,    % ±
%		"003D,    % =
%		"2260,    % ≠
%		"2248,    % ≈
%		"003C,    % <
%		"003E,    % >
%		"2264,    % ≤
%		"2265,    % ≥
%		"2202,    % ∂
%		\partial, % s. https://github.com/wspr/unicode-math/issues/416
%		"221E,    % ∞
%		"2020,    % †
%		"2021,    % ‡
%		"0025,    % %
%		"2030,    % ‰ (aber s. https://github.com/wspr/unicode-math/issues/452)
%		"0021,    % !
%		"003F,    % ?
%		"002E,    % .
%		"002C,    % ,
%		"003A,    % :
%		"003B,    % ;
%		"0026,    % &
%		"0023,    % #
%		"0040,    % @
%		"00A7,    % §
%		"20AC,    % €
%		"0024,    % $
%		"00A3,    % £
%		"00A5,    % ¥
%	}
%]

% Offizielles WWU-LaTeX-Paket für Präsentation (leicht modifiziert)
% Mögliche Optionen:
% - english: Verwendet englischen Claim („living.knowledge“)
% - Verschiedene Farbvarianten:
%   pantoneblack7, pantone312, pantone7462, pantone3135, pantone315, pantone369,
%   pantone390, pantone396, pantone3282, pantoneprozessyellow
% - Verschiedene Titel-Motive:
%   - belltower (Standard-Wert): Glockenturm des Schlosses
%   - wedge: Textkeil
%   - prinz: WWU-Schriftzug auf dem Prinzipalmarkt (Foto)
% - inverse: Inverses Titelbild (Weiß auf farbigem Hintergrund statt Farbe auf
%            weißem Hintergrund)
% - wide: Seitenverhältnis 16:10 verwenden (statt Standardwert 4:3)
\usepackage[english,wide,pantone312,inverse,wedge]{wwustyle-mod}


%%%%%%%%%%%%%%%%%%%%%%%%%
% METROPOLIS   TEMPLATE %
%%%%%%%%%%%%%%%%%%%%%%%%%
%\usetheme{metropolis} 


%\usepackage[wide,pantone312,inverse]{wwustyle}
% Typographische Verbesserungen (Verbot mancher Ligaturen)
\usepackage{selnolig}
% Typographische Verbesserungen (Mikrotypographie)
\usepackage{microtype}

% Daten/Zeiten formatieren
\usepackage[useregional]{datetime2}
% Formatierung von Telefonnummern
\usepackage{phonenumbers}
% „Schöne“ Brüche im Fließtext mit \sfrac
\usepackage{xfrac}
% Ermöglicht die Nutzung von „\SI{Zahl}{Einheit}“
\usepackage{siunitx}
% Automatisches Umwandeln von Anführungszeichen
\usepackage{csquotes}

% Farben ermöglichen
\usepackage{xcolor}
% Paket für Bilder-Einbindung (EPS, PNG, JPG, PDF)
\usepackage{graphicx}
% .tex-Dateien mit \includegraphics einbinden
\usepackage{gincltex}
% Bessere Verarbeitung von Dateinamen für \includegraphics etc.
\usepackage{grffile}

% latex
\renewcommand{\arraystretch}{1.3}
% graphicx
% Standardmäßig „keepaspectratio“ verwenden
% s. https://tex.stackexchange.com/a/91619/51235
\setkeys{Gin}{keepaspectratio}
% hyperref
\hypersetup{unicode}
% siunitx
\sisetup{
	locale=US,
	binary-units,
	quotient-mode=fraction,
	per-mode=fraction,
	fraction-function=\sfrac,
	detect-weight
}
\sisetup{detect-all, math-rm = \ensuremath, math-sf = \ensuremath}
% csquotes
\MakeOuterQuote{"}

%\institutelogo{\raisebox{-5.75mm}{\includegraphics[width=3.8cm]{fsphys-logo.pdf}}}
%\institutelogosmall{\raisebox{-2.5mm}[0pt][0pt]{\includegraphics[width=2.6cm]{fsphys-logo.pdf}}}

%%%%%%%%%%%%%%%%%%%%%%%%%%%%%%%%%%%%%%%

% Zusätzliche Einstellungen/Befehle
\let\strong\textbf
\newcommand{\email}[1]{\href{mailto:#1}{\texttt{#1}}}
\newfontfamily\DejaSans{DejaVu Sans}

\usepackage{nameref}
\makeatletter
\newcommand*{\currentname}{\@currentlabelname}
\makeatother

\newenvironment{nframe}{
	\begin{frame}[noframenumbering]
	\frametitle{\subsecname}
}{\end{frame}}

\newcommand{\foo}{\makebox[0pt]{\textbullet}\hskip-0.5pt\vrule width 1pt\hspace{\labelsep}}


%\usepackage{chronology}

\usepackage{luacode}
\usepackage{tikz}

\usetikzlibrary{graphdrawing,arrows, calc, math, decorations.markings, positioning}
%FIX LUA https://tex.stackexchange.com/questions/453132/fresh-install-of-tl2018-no-tikz-graph-drawing-libraries-found
\begin{luacode*}
	function pgf_lookup_and_require(name)
	local sep = package.config:sub(1,1)
	local function lookup(name)
	local sub = name:gsub('%.',sep)
	if kpse.find_file(sub, 'lua') then
	require(name)
	elseif kpse.find_file(sub, 'clua') then
	collectgarbage('stop')
	require(name)
	collectgarbage('restart')
	else
	return false
	end
	return true
	end
	return
	lookup('pgf.gd.' .. name .. '.library') or
	lookup('pgf.gd.' .. name) or
	lookup(name .. '.library') or
	lookup(name)
	end
\end{luacode*}
\usepackage[compat=1.1.0]{tikz-feynman}
\usepackage{svg}

\usepackage{xspace} 

\usepackage[backend=biber,sorting=none,style=nature]{biblatex}
\usepackage[font=small]{caption}
\renewcommand{\footnotesize}{\scriptsize}
\AtEveryCitekey{\clearfield{title}\clearfield{doi}}




\usepackage{amsmath}
\usepackage{amsfonts}
\usepackage{amssymb}
    \newcommand{\bra}[1]{\ensuremath{\left\langle#1\right|}}
\newcommand{\ket}[1]{\ensuremath{\left|#1\right\rangle}}
\newcommand{\bracket}[2]{\ensuremath{\left\langle #1 \middle| #2 \right\rangle}}
\newcommand{\matrixel}[3]{\ensuremath{\left\langle #1 \middle| #2 \middle| #3 \right\rangle}}

\usepackage[absolute,overlay]{textpos}
\usepackage{ifthen}

\usepackage{slashed}
%\usepackage{pgfpages}
% Abbildungen nebeneinander (subfigure, subtable)
\usepackage{subcaption}
%\usepackage{minted}

%\catcode\ß=13
%\defß{\ss}
%\numberwithin{equation}{subsection}

%\setbeameroption{show only notes}
\usepackage{xfp}
\usepackage{import}
\usetikzlibrary{calc}
\usetikzlibrary{external}


\usepackage{xcolor}
\usepackage{contour}
\usepackage{mathtools}
\usepackage{physics}
%\PassOptionsToPackage{export}{adjustbox}

\renewcommand{\i}{{\mathrm{i}}}
\newcommand{\defeq}{\vcentcolon=}
\newcommand{\eqdef}{=\vcentcolon}

\usepackage[makeroom]{cancel}


\usepackage{color, colortbl,booktabs}
