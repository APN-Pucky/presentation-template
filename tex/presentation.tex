% !TeX program = lualatex
% NOTE: This file must be run using xelatex or lualatex!

% different aspect ratios can be set using the option “aspectratio”,
% e.g. “aspectratio=169” for 16:9
\documentclass[english]{beamer}
% language setting
\usepackage{polyglossia}
\setmainlanguage{english}

% Official WWU LaTeX package for presentations: wwustyle
% The use of wwustyle requires xelatex or lualatex (due to the fonts).
% If you have any questions regarding this, contact fsphys@uni-muenster.de
% Available Options:
% - english: uses the English claim (“living.knowledge”)
% - Color variants:
%   pantone312, pantone3135, pantone7462,
%   pantoneblack7, pantone315, pantone369, pantone390
% - Different title images:
%   - belltower (default): castle bell tower
%   - wedge: text wedge
%   - prinz: WWU letters at Prinzipalmarkt (photo)
% - inverse: inverse title image (white on colored background
%            instead of color on white background)
% - nopagenumbering:   disables slide numbering
% - fullpagenumbering: displays total slide number in addition to current slide number
\usepackage[english, pantone312]{wwustyle}

% custom settings can be placed in this file
%%% settings for the correct use of wwustyle.sty
% Einstellungen der Schriftart (Meta Office Pro) für Text und Mathematik
% math**=sym angeben, damit auch diese Befehle die Schriftart Meta verwenden
\usepackage[mathrm=sym, mathit=sym, mathsf=sym, mathbf=sym]{unicode-math}
\setmainfont{MetaOffcPro}[Path=fonts/,
	Extension=.ttf,
	UprightFont=*-Norm,
	UprightFeatures={
		SmallCapsFont={MetaScOffcPro-Norm}
	},
	ItalicFont=*-NormIta,
	ItalicFeatures={
		SmallCapsFont={MetaScOffcPro-NormIta}
	},
	BoldFont=*-Bold,
	BoldFeatures={
		SmallCapsFont={MetaScOffcPro-Bold}
	},
	BoldItalicFont=*-Bold,
	BoldItalicFeatures={FakeSlant},
]
\setsansfont{MetaOffcPro}[Path=fonts/,
	Extension=.ttf,
	UprightFont=*-Norm,
	UprightFeatures={
		SmallCapsFont={MetaScOffcPro-Norm}
	},
	ItalicFont=*-NormIta,
	ItalicFeatures={
		SmallCapsFont={MetaScOffcPro-NormIta}
	},
	BoldFont=*-Bold,
	BoldFeatures={
		SmallCapsFont={MetaScOffcPro-Bold}
	},
	BoldItalicFont=*-Bold,
	BoldItalicFeatures={FakeSlant},
]

\setmathfont{Latin Modern Math}
% Meta Office Pro für die Bereiche nutzen, für die Glyphen existieren
\setmathfont{MetaOffcPro-Norm.ttf}[Path=fonts/,
	range=up/{greek,Greek,latin,Latin,num}]
\setmathfont{MetaOffcPro-NormIta.ttf}[Path=fonts/,
	range=it/{greek,Greek,latin,Latin,num}]
\setmathfont{MetaOffcPro-Bold.ttf}[Path=fonts/,
	range=bfup/{greek,Greek,latin,Latin,num}]
\setmathfont{MetaOffcPro-Bold.ttf}[Path=fonts/, UprightFeatures={FakeSlant},
	range=bfit/{greek,Greek,latin,Latin,num}]
% Symbole (leider enthält Meta Office Pro nicht das Symbol ∓)
\setmathfont{MetaOffcPro-Norm.ttf}[Path=fonts/,
	range={`\+, `\-, `\×, `\÷, `\⋅, `\*, `\/, `\⁄, `\±,
		`\=, `\≠, `\≈, `\<, `\>, `\≤, `\≥, \partial, `\∞, `\†, `\‡,
		`\%, `\‰, `\!, `\?, `\., `\,, `\:, `\;, `\&, `\#, `\@,
		`\§, `\€, `\$, `\£, `\¥, `\©, `\®,
	}
]


%% =============== load packages ================================
% typographical improvements (micro-typography)
\usepackage{microtype}

% enable colors
\usepackage{xcolor}
% package for image inclusion (EPS, PNG, JPG, PDF)
\usepackage{graphicx}
% include .tex files with \includegraphics
\usepackage{gincltex}
% better handling of filenames with \includegraphics etc.
\usepackage{grffile}

%% =============== package settings =============================
% LaTeX
\renewcommand{\arraystretch}{1.3}
% graphicx
% use “keepaspectratio” as a default
% s. https://tex.stackexchange.com/a/91619
\setkeys{Gin}{keepaspectratio}
% hyperref
\hypersetup{unicode}

%% =============== additional settings/commands =================
\newcommand*{\email}[1]{\href{mailto:#1}{\texttt{#1}}}


% with \institutelogo, the logo on the title slide can be set
\institutelogo{\parbox{4cm}{\footnotesize space for secondary logo(s)/\\name of department, institute or SFB}}
% with \institutelogosmall, the logo in the slide headline can be set
\institutelogosmall{“institutelogosmall” appears here}

\title[Here is the presentation title]{Here is the presentation title on two separate lines}
\subtitle{Subtitle or short description of the presentation}
\author{Presenter's name}
\date{\today}
% \subject und \institute are not directly used by the template
%\subject{Subject}
%\institute{Institute}
% Keywords for metadata in the PDF file
%\keywords{...}

% The commands commented here place a bar in the main color
% behind the white text. This can make sense if the title image
% “Prinzipalmarkt” has been chose using the option “prinz”.
%\title[Here is the presentation title]{\bgbox{Here is the presentation title}\\\bgbox{on two separate lines}}
%\subtitle{\bgbox{Subtitle or short description of the presentation}}
%\author{\bgbox{Presenter's name}}
%\date{\bgbox{Date}}

\begin{document}

% title slide
\begin{frame}[plain, noframenumbering]
	\titlepage
\end{frame}

\begin{frame}
	\frametitle{Here is a title which takes up two lines due to its length}

	Here is some \textbf{dummy text} which will be replaced by the real text later.
	In this \textbf{line, there is some dummy text}.
	Here is some dummy text – it will be replaced later.
	In this \textbf{line, there is some dummy text}.
	Here is dummy text.
\end{frame}

\begin{frame}
	\frametitle{Here is a one-line title}
	\framesubtitle{With subtitle}

	First level (text)
	\begin{itemize}
		\item \textbf{Second level} (list)
		\begin{itemize}
			\item \textbf{Third level} (list)
			\begin{itemize}
				\item \textbf{Fourth level} (list)
			\end{itemize}
		\end{itemize}
	\end{itemize}
\end{frame}

\begin{frame}
	\frametitle{Slide title}

	Here is some text!

	\begin{block}{A “normal” block}
		Contents here.
	\end{block}

	\texttt{itemize} and \texttt{enumerate}:
	\begin{itemize}
		\item An item
		\begin{itemize}
			\item A sub-item
		\end{itemize}
		\item Another item
	\end{itemize}
	\begin{enumerate}
		\item An item
		\begin{enumerate}
			\item A sub-item
		\end{enumerate}
		\item Another item
	\end{enumerate}
\end{frame}

\begin{frame}
	\frametitle{An alert block}

	\begin{alertblock}{Attention!}
		Red comes into play here!
	\end{alertblock}
\end{frame}

\begin{frame}
	\frametitle{An example block}

	\begin{exampleblock}{Example}
		Green comes into play here!
	\end{exampleblock}
\end{frame}

\begin{frame}
	\frametitle{More lists}

	Nested lists
	\begin{itemize}
		\item \textbf{First level}
		\begin{itemize}
			\item \textbf{Second level}
			\begin{itemize}
				\item \textbf{Third level}
				\item \textbf{Third level}
			\end{itemize}
		\end{itemize}
	\end{itemize}
	Nested lists
	\begin{itemize}
		\item \textbf{First level}
		\begin{itemize}
			\item \textbf{Second level}
			\begin{itemize}
				\item \textbf{Third level} with text behind
				\item \textbf{Third level}

				with text below
			\end{itemize}
		\end{itemize}
	\end{itemize}
\end{frame}

\end{document}
