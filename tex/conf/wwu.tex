%%% Einstellungen zur richtigen Benutzung von wwustyle.sty
\usefonttheme{professionalfonts}
%\RequirePackage{wwustyle/meta-office-pro}
% Einstellungen der Schriftart (Meta Office Pro) für Text und Mathematik
% math**=sym angeben, damit auch diese Befehle die Schriftart Meta verwenden
\usepackage[mathrm=sym, mathit=sym, mathsf=sym, mathbf=sym]{unicode-math}
\setmainfont{MetaOffcPro}[Path=fonts/,
Extension=.ttf,
UprightFont=*-Norm,
UprightFeatures={
	SmallCapsFont={MetaScOffcPro-Norm}
},
ItalicFont=*-Norm,
ItalicFeatures={
	SmallCapsFont={MetaScOffcPro-NormIta}
},
BoldFont=*-Bold,
BoldFeatures={
	SmallCapsFont={MetaScOffcPro-Bold}
},
BoldItalicFont=*-Bold,
BoldItalicFeatures={FakeSlant},
]
\setsansfont{MetaOffcPro}[Path=fonts/,
Extension=.ttf,
UprightFont=*-Norm,
UprightFeatures={
	SmallCapsFont={MetaScOffcPro-Norm}
},
ItalicFont=*-Norm,
ItalicFeatures={
	SmallCapsFont={MetaScOffcPro-NormIta}
},
BoldFont=*-Bold,
BoldFeatures={
	SmallCapsFont={MetaScOffcPro-Bold}
},
BoldItalicFont=*-Bold,
BoldItalicFeatures={FakeSlant},
]
%\setmathfont{Latin Modern Math}
%% Meta Office Pro für die Bereiche nutzen, für die Glyphen existieren
%\setmathfont{MetaOffcPro-Norm.ttf}[Path=fonts/,
%range=up/{greek,Greek,latin,Latin,num}]
%\setmathfont{MetaOffcPro-NormIta.ttf}[Path=fonts/,
%range=it/{greek,Greek,latin,Latin,num}]
%\setmathfont{MetaOffcPro-Bold.ttf}[Path=fonts/,
%range=bfup/{greek,Greek,latin,Latin,num}]
%\setmathfont{MetaOffcPro-Bold.ttf}[Path=fonts/, UprightFeatures={FakeSlant},
%range=bfit/{greek,Greek,latin,Latin,num}]
%% Symbole (leider enthält Meta Office Pro nicht das Symbol ∓)
%%\setmathfont{MetaOffcPro-Norm.ttf}[Path=fonts/]
%
%\setmathfont{MetaOffcPro-Norm.ttf}[Path=fonts/,
%	range={
%		"002B,    % +
%		"002D,    % -
%		"2212,    % −
%		\minus,   % s. https://github.com/wspr/unicode-math/issues/451
%		"00D7,    % ×
%		"00F7,    % ÷
%		"00B7,    % ·
%		"22C5,    % ⋅
%		"002F,    % /
%		"2044,    % ⁄
%		"00B1,    % ±
%		"003D,    % =
%		"2260,    % ≠
%		"2248,    % ≈
%		"003C,    % <
%		"003E,    % >
%		"2264,    % ≤
%		"2265,    % ≥
%		"2202,    % ∂
%		\partial, % s. https://github.com/wspr/unicode-math/issues/416
%		"221E,    % ∞
%		"2020,    % †
%		"2021,    % ‡
%		"0025,    % %
%		"2030,    % ‰ (aber s. https://github.com/wspr/unicode-math/issues/452)
%		"0021,    % !
%		"003F,    % ?
%		"002E,    % .
%		"002C,    % ,
%		"003A,    % :
%		"003B,    % ;
%		"0026,    % &
%		"0023,    % #
%		"0040,    % @
%		"00A7,    % §
%		"20AC,    % €
%		"0024,    % $
%		"00A3,    % £
%		"00A5,    % ¥
%	}
%]

% Offizielles WWU-LaTeX-Paket für Präsentation (leicht modifiziert)
% Mögliche Optionen:
% - english: Verwendet englischen Claim („living.knowledge“)
% - Verschiedene Farbvarianten:
%   pantoneblack7, pantone312, pantone7462, pantone3135, pantone315, pantone369,
%   pantone390, pantone396, pantone3282, pantoneprozessyellow
% - Verschiedene Titel-Motive:
%   - belltower (Standard-Wert): Glockenturm des Schlosses
%   - wedge: Textkeil
%   - prinz: WWU-Schriftzug auf dem Prinzipalmarkt (Foto)
% - inverse: Inverses Titelbild (Weiß auf farbigem Hintergrund statt Farbe auf
%            weißem Hintergrund)
% - wide: Seitenverhältnis 16:10 verwenden (statt Standardwert 4:3)
\usepackage[english,wide,pantone312,inverse,wedge]{wwustyle-mod}