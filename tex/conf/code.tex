
\newif\iflongb
\longbfalse

\newcommand{\iflong}[1]{\iflongb #1\else \fi}
\newcommand{\ifnotlong}[1]{\iflongb \else #1 \fi}


\newcommand{\ifger}[2]{#2}

\def \speeditup {1}
\newcommand{\speed}[1]{%
  \ifthenelse{0=\speeditup}{#1}{}
}
\newcommand{\isplit}[3][0.48]{
\begin{columns}[T]
  \begin{column}{#1\linewidth}
    #2
\end{column}
  \begin{column}{\fpeval{0.98-#1}\linewidth}
    #3
\end{column}
\end{columns}
}
\newcommand{\ilsplit}[3][0.48]{
\begin{columns}[T]
  \begin{column}{#1\linewidth}
    #2
\end{column}
%\hspace{-40pt}
\vrule{}
\hspace{10pt}
  \begin{column}{\fpeval{0.98-#1}\linewidth}
    #3
\end{column}
\end{columns}
}

\newenvironment{iframe}[1][0pt]{
  \begin{frame}[c]{\centering \subsecname}
    \vspace{#1}
}{\end{frame}}

\newenvironment{iiframe}{
	\begin{frame}[t]{\subsecname}
  %\frametitle{\subsecname}
  %\vspace{-25pt}

}{\end{frame}}


\newenvironment{iequation}{
	\begin{equation*}
}{\end{equation*}}

\newcommand{\inote}[1]{
\note[item]{#1}
}



\newcommand{\ifigure}[2][1]{
\begin{figure}[H]
\includegraphics[width=#1\linewidth]{#2}
%\includegraphics[height=\fpeval{0.8*#1}\textheight]{#2}
\end{figure}
}


%\newcommand{\ifeyn}[2][]{
%    \begin{tikzpicture}[baseline=-0.5ex]
%        \begin{feynman}[inline,node distance=1.4cm]
%          \vertex[blob,anchor=base] (c) at ( 0, 0) #1;
%          \vertex[above left =of c] (i1);
%          \vertex[below left = of c] (i2)  ;
%          \vertex[above right = of c] (f1)  ;
%          \vertex[below right = of c] (f3)  ;
%          \vertex (f2)  at ($(f1)!0.5!(f3)$);
%          \vertex[above = 0.75cm of c] (va);
%          \vertex[below= 0.75cm of c] (vb);
%          \vertex[left = 0.50cm of c] (ha);
%          \vertex[right= 0.50cm of c] (hb);
%          \vertex (vaf1) at ($(va)!0.5!(f1)$);
%          \vertex (vbf3) at ($(vb)!0.5!(f3)$);
%          \vertex (cf1) at ($(c)!0.5!(f1)$);
%          \vertex (hbf1) at ($(hb)!0.5!(f1)$);
%          \vertex (chb) at ($(c)!0.5!(hb)$);
%          \vertex (chbf1) at ($(chb)!0.5!(f1)$);
%          \vertex (i1c) at ($(i1)!0.5!(c)$);
%          \vertex[right = of i1] (ri1);
%          #2
%      \end{feynman}
%    \end{tikzpicture}
%}
%\newenvironment{ifeyn}[1][]{
%    \begin{tikzpicture}[baseline=-0.5ex]
%        \begin{feynman}[inline,node distance=1.4cm]
%          \vertex[blob,anchor=base] (c) at ( 0, 0) #1;
%          \vertex[above left =of c] (i1);
%          \vertex[below left = of c] (i2)  ;
%          \vertex[above right = of c] (f1)  ;
%          \vertex[below right = of c] (f3)  ;
%          \vertex (f2)  at ($(f1)!0.5!(f3)$);
%          \vertex[above = 0.75cm of c] (va);
%          \vertex[below= 0.75cm of c] (vb);
%          \vertex[left = 0.50cm of c] (ha);
%          \vertex[right= 0.50cm of c] (hb);
%          \vertex (vaf1) at ($(va)!0.5!(f1)$);
%          \vertex (vbf3) at ($(vb)!0.5!(f3)$);
%          \vertex (cf1) at ($(c)!0.5!(f1)$);
%          \vertex (hbf1) at ($(hb)!0.5!(f1)$);
%          \vertex (chb) at ($(c)!0.5!(hb)$);
%          \vertex (chbf1) at ($(chb)!0.5!(f1)$);
%          \vertex (i1c) at ($(i1)!0.5!(c)$);
%          \vertex[right = of i1] (ri1);
%        }{
%      \end{feynman}
%      \end{tikzpicture}
%}
\tikzfeynmanset{every blob={/tikz/pattern color={gray!80}}}

\tikzstyle arrowstyle=[scale=1]
\tikzstyle directed=[postaction={decorate,decoration={markings,
    mark=at position .65 with {\arrow[arrowstyle]{stealth}}}}]

\tikzstyle enddirected=[postaction={decorate,decoration={markings,
    mark=at position 1.0 with {\arrow[arrowstyle]{stealth}}}}]
\tikzstyle reverse directed=[postaction={decorate,decoration={markings,
    mark=at position .65 with {\arrowreversed[arrowstyle]{stealth};}}}]

\newcommand{\avg}[1] {
  \langle #1 \rangle
}

\newcommand{\prog}[1]{\texttt{#1}}
\DeclareMathOperator{\sign}{sign}

\newcommand{\gluon}{g}
\newcommand{\quark}{q}
\newcommand{\squark}{{\tilde q}}
\newcommand{\upsquark}{{\tilde u}}
\newcommand{\downsquark}{{\tilde d}}
\newcommand{\gluino}{{\tilde g}}
\newcommand{\gaugino}{{\tilde \chi}}
\newcommand{\ghost}{c}

\definecolor{Gray}{gray}{0.9}

\makeatletter
\DeclareCiteCommand{\arxivcite}[\mkbibbrackets]
  {\usebibmacro{prenote}}
  {\usebibmacro{citeindex}%
   \newunit
   \usebibmacro{eprint}}
  {\multicitedelim}
  {\usebibmacro{postnote}}
  \makeatother