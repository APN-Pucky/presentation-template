% Option          Farbe/Hintergrundbild
% -------------------------------------
% pantone312        Hellblau
% pantone315        Dunkelblau
% pantone3282       Dunkelgruen
% prinz             Prinzipalmarkt
% wedge             Textkeil rechts
% belltower         Glockenturm des Schlosses (Voreinstellung)
%
% wide              breites Pr�sentationsformat
% Farbe und Hintergrund sind beliebig kombinierbar.
\documentclass[inverse, prinz]{beamer}
\usepackage{wwustyle}
\usepackage[ngerman]{babel}
\usepackage[utf8]{inputenc}
\usepackage[T1]{fontenc}

% Uncomment the following two lines if you want to prepare your document for
% the fast mode.
% \usetikzlibrary{external}
% \tikzexternalize

\author{\bgbox{Name: der Referentin / des Referenten}}
\title{Hier steht der Titel\\ der Präsentation}
% Der hier auskommentierte \title{} unterlegt die wei�e Schrift
% mit einem Balken in der Hauptfarbe des Dokuments. Dies kann sinnvoll
% sein, wenn mit der Option �prinz' der Prinzipalmarkt als
% Hintergrundbild ausgew�hlt wurde. Der Befehl \bgbox{} entspricht
% von der Funktion her \mbox{}, ist aber lediglich auf der Titelseite
% der Pr�sentation aktiviert.
\title{\bgbox{Hier steht der Titel} \\\bgbox{der Präsentation}}
\subtitle{\bgbox{Optionen für \texttt{\string\documentclass}: inverse, wedge, Verwendung von \texttt{\string\bgbox}}}

\institutelogo{\vbox{\hsize=35mm Freiraum für Sekundärlogo(s) / Name \\ des Fachbereichs, Instituts oder SFBs}}



\begin{document}

%%%%%%%%%%%%%%% WWUstyle "fast" mode %%%%%%%%%%%%%%%%%%%%%
% Do the following steps in order to speed up the compilation time of your
% presentation:
%
% 1. Include the externalization tikz library in the preamble of your document.
%    This is always recommended if you are using tikz in your document.
% 2. Uncomment the \wwupreparefastmode command below
% 3. Compile your document with command line option '-shell-escape',
%    e.g.: 'pdflatex -shell-escape beamer.tex'
% 4. Comment (or delete) the \wwupreparefastmode
% 5. Add option 'fast' to the 'wwustyle' package declaration line.
% 6. Be happy!

% \wwupreparefastmode


\begin{frame}[plain]
  \maketitle
\end{frame}

\begin{frame}
  \frametitle{Folientitel sollten kurz und pr\"agnant sein}
  {Gestufte Listen}
  \pause
  \begin{itemize}
    \item \textbf{Erste Ebene}
      \begin{itemize}
        \item \textbf{Zweite Ebene}
          \begin{itemize}
            \item \textbf{Dritte Ebene}
          \end{itemize}
      \end{itemize}
  \end{itemize}
  \vfill
\end{frame}

\begin{frame}
  \frametitle{Folientitel sollten kurz und pr\"agnant sein}
  \begin{block}{Hervorhebungen}
    \textbf{Wenn man Dinge hervorheben m\"ochte, nutzt man entweder Fettdruck,}
    \textit{kursive Schrift} \alert{ oder das Schl\"usselwort ``alert''}.
  Auch ``enumerate''-Umgebungen werden von der Stilvorlage überschrieben:
  \end{block}
  \pause
  \begin{enumerate}
    \item So wird sichergestellt,
        \item dass alle Elemente der Präsentation
            \item dieselbe Farbe nutzen.
  \end{enumerate}
\end{frame}

% \begin{frame}
%   \frametitle{Ein Alerted-Block}
%   \begin{alertblock}{Achtung!}
%     Hier kommt Rot ins Spiel!
%   \end{alertblock}
% \end{frame}
%
% \begin{frame}
%   \frametitle{Ein Example-Block}
%   \begin{exampleblock}{Beispiel}
%     Hier kommt Grün ins Spiel!
%   \end{exampleblock}
% \end{frame}

\end{document}
